% This file is part of Simple Parser.
%
% Simple Parser is free software: you can redistribute it and/or modify
% it under the terms of the GNU Lesser General Public License as published
% by the Free Software Foundation, either version 3 of the License, or
% (at your option) any later version.
%
% Simple Parser is distributed in the hope that it will be useful,
% but WITHOUT ANY WARRANTY; without even the implied warranty of
% MERCHANTABILITY or FITNESS FOR A PARTICULAR PURPOSE.  See the
% GNU Lesser General Public License for more details.
%
% You should have received a copy of the GNU Lesser General Public License
% along with Simple Parser.  If not, see <http://www.gnu.org/licenses/>.

\section{Introduction}

SP (Simple Parser) is a Python\footnote{Python is a wonderful object oriented programming language available at \url{http://www.python.org}} parser generator.
It is aimed at easy usage rather than performance.

\section{License}

SP is available under the GNU Lesser General Public.

\begin{quote}
Simple Parser: A Python parser generator

Copyright (C) 2009 Christophe Delord
 
This library is free software; you can redistribute it and/or
modify it under the terms of the GNU Lesser General Public
License as published by the Free Software Foundation; either
version 2.1 of the License, or (at your option) any later version.

This library is distributed in the hope that it will be useful,
but WITHOUT ANY WARRANTY; without even the implied warranty of
MERCHANTABILITY or FITNESS FOR A PARTICULAR PURPOSE.  See the GNU
Lesser General Public License for more details.

You should have received a copy of the GNU Lesser General Public
License along with this library; if not, write to the Free Software
Foundation, Inc., 59 Temple Place, Suite 330, Boston, MA  02111-1307  USA 
\end{quote}

\section{Structure of the document}

\begin{description}
\item [Part~\ref{sp:intro}]
starts smoothly with a gentle tutorial as an introduction.
I think this tutorial may be sufficent to start with SP.
\item [Part~\ref{sp:core}]
is a reference documentation. It will detail SP as much as possible.
\item [Part~\ref{sp:examples}]
gives the reader some examples to illustrate SP.
\end{description}
