% This file is part of Simple Parser.
%
% Simple Parser is free software: you can redistribute it and/or modify
% it under the terms of the GNU Lesser General Public License as published
% by the Free Software Foundation, either version 3 of the License, or
% (at your option) any later version.
%
% Simple Parser is distributed in the hope that it will be useful,
% but WITHOUT ANY WARRANTY; without even the implied warranty of
% MERCHANTABILITY or FITNESS FOR A PARTICULAR PURPOSE.  See the
% GNU Lesser General Public License for more details.
%
% You should have received a copy of the GNU Lesser General Public License
% along with Simple Parser.  If not, see <http://www.gnu.org/licenses/>.

\section{Introduction}

This chapter presents an extention of the calculator described in the tutorial (see~\ref{sp:tutorial}).
This calculator has more functions and a memory.

\section{New functions}

\subsection{Memories}

The calculator has memories.
A memory cell is identified by a name.
For example, if the user types \emph{$pi = 3.14$}, the memory cell named \emph{pi} will contain the value of \emph{$\pi$} and \emph{$2*pi$} will return \emph{$6.28$}.

The variables are saved in a dictionnary.

\section{Source code}

Here is the complete source code (\emph{calc.py}):

\verbatimtabinput[4]{../examples/calc.py}
