% This file is part of Simple Parser.
%
% Simple Parser is free software: you can redistribute it and/or modify
% it under the terms of the GNU Lesser General Public License as published
% by the Free Software Foundation, either version 3 of the License, or
% (at your option) any later version.
%
% Simple Parser is distributed in the hope that it will be useful,
% but WITHOUT ANY WARRANTY; without even the implied warranty of
% MERCHANTABILITY or FITNESS FOR A PARTICULAR PURPOSE.  See the
% GNU Lesser General Public License for more details.
%
% You should have received a copy of the GNU Lesser General Public License
% along with Simple Parser.  If not, see <http://www.gnu.org/licenses/>.

\section{SP grammar structure}

SP grammars are Python objects.
SP grammars may contain two parts:

\begin{description}
    \item [Tokens]
        are built by the \emph{R} or \emph{K} keywords (see~\ref{lexer:token_def}).
    \item [Rules]
        are described after tokens (see~\ref{grammar:struct}) in a \emph{Separator context}.
\end{description}

See figure~\ref{grammar:struct} for a generic SP grammar.

\begin{code}
\caption{SP grammar structure}                             \label{grammar:struct}
\begin{verbatimtab}[4]
def Foo():

    # Tokens
    number = R(r'\d+') / int

    # Rules
    with Separator(r'\s+'):
        S = number[:]

    return S

foo = Foo()
result = foo("42 43 44") # return [42, 43, 44]
\end{verbatimtab}
\end{code}

\section{Comments}

Comments in SP start with \emph{\#} and run until the end of the line, as in Python.

\begin{verbatimtab}[4]
    # This is a comment
\end{verbatimtab}

